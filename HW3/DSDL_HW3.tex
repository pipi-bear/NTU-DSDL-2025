\documentclass{article}
\usepackage[utf8]{inputenc}
\usepackage[english]{babel}
\usepackage[]{amsthm} 
\usepackage[]{amssymb} 
\usepackage{amsmath}
\usepackage{hyperref}
\usepackage{multirow}
\usepackage{tabularx}
\usepackage{graphicx}
\graphicspath{ {./solution_images/} }
\usepackage{float}
\usepackage[table]{xcolor}
\usepackage[dvipsnames]{xcolor}
\usepackage[thinc]{esdiff}


\title{Digital System Design and Lab: HW3}
\author{Lo Chun, Chou \\ R13922136}
\date\today

\begin{document}
\setlength{\parindent}{0pt}
\maketitle 

\section*{1}

By lecture slide LEC-09 p.10-11, 
we knew that using two three-state buffers with one inverter could do data selection, 
and is equivalent to a 2-to-1 MUX:

\begin{figure}[h]
    \centering
    \includegraphics[width=0.5\textwidth]{1_three_state_buffer_w_inverter.jpeg}
\end{figure}

So, we can use the two 4-to-1 MUXs and this setting to implement the 8-to-1 MUX as follows:

\begin{figure}[h]
    \centering
    \includegraphics[width=0.5\textwidth]{1_sol.jpeg}
\end{figure}

\section*{2}

First, observe that part of the given circuit is a $\bar{S}-\bar{R}$ latch:

\begin{figure}[h]
    \centering
    \includegraphics[width=0.8\textwidth]{2_Sbar_Rbar_latch.jpeg}
\end{figure}

with the inputs:

\begin{align*}
    \bar{S} &= (AB)' = A' + B' \\
    \bar{R} &= (AB')' = A' + B
\end{align*}

Thus, we can form the truth table of this latch by checking the values of $\bar{S}$, $\bar{R}$ and $Q$ in the table above:

\begin{center}
    \begin{tabular}{ |c|c||c|c|c||c| } 
        \hline
        \cline{1-6}
        A & B & $\raisebox{-0.4ex}{$\bar{S}$}$ & $\raisebox{-0.4ex}{$\bar{R}$}$ & Q & $Q^+$ \\
        \hline
        0 & 0 & 1 & 1 & 0 & 0 \\
        0 & 0 & 1 & 1 & 1 & 1 \\
        0 & 1 & 1 & 1 & 0 & 0 \\
        0 & 1 & 1 & 1 & 1 & 1 \\
        1 & 0 & 1 & 0 & 0 & 0 \\
        1 & 0 & 1 & 0 & 1 & 0 \\
        1 & 1 & 0 & 1 & 0 & 1 \\
        1 & 1 & 0 & 1 & 1 & 1 \\
        \hline
    \end{tabular}
\end{center}



\section*{3}

The following is the different cases of the latch:

\begin{figure}[H]
    \centering
    \includegraphics[width=0.7\textwidth]{3_derivation.jpeg}
\end{figure}

\subsection*{(1)}

From the above cases, we can see that when $R = 1$ and $H = 0$, $P = 1 \neq Q' = 0$.
Therefore, we should not let:

\begin{align*}
    R = 1 \quad \text{and} \quad H = 0
\end{align*}

\subsection*{(2)}

The next-state table is shown below:

\begin{center}
    \begin{tabular}{ |c|c|c||c| } 
        \hline
        \cline{1-4}
        R & H & Q & Q+ \\ 
        \hline
        0 & 0 & 0 & 0 \\ 
        0 & 0 & 1 & 0 \\ 
        0 & 1 & 0 & 0 \\ 
        0 & 1 & 1 & 1 \\ 
        1 & 0 & 0 & X \\
        1 & 0 & 1 & 1 \\ 
        1 & 1 & 0 & X \\ 
        1 & 1 & 1 & 1 \\ 
        \hline
    \end{tabular}
\end{center}

And we can construct the K-map as follows:

\begin{figure}[H]
    \centering
    \includegraphics[width=0.4\textwidth]{3_kmap.jpeg}
\end{figure}

Which would give us the characteristic equation:

\begin{align*}
    Q^+ = R + H \cdot Q
\end{align*}

\section*{4}

\subsection*{(1)}

\begin{figure}[H]
    \centering
    \includegraphics[width=1\textwidth]{4_1_clagl.png}
\end{figure}

\subsection*{(2)}

\begin{figure}[H]
    \centering
    \includegraphics[width=0.9\textwidth]{4_2_rcagl.png}
\end{figure}


\section*{5}

I use Surfer instead of GTKWave to present the waveform:

\begin{figure}[H]
    \centering
    \includegraphics[width=1\textwidth]{5_waveform.png}
\end{figure}

\section*{6}

We can find the maximum delay and one of the transition from the attached terminal output screenshots.

\subsection*{(1)}

\begin{figure}[H]
    \centering
    \includegraphics[width=0.7\textwidth]{6_(1).png}
\end{figure}

\subsection*{(2)}

\begin{figure}[H]
    \centering
    \includegraphics[width=0.7\textwidth]{6_(2).png}
\end{figure}

\section*{7}

First, since we are assuming a $n$-bit carry lookahead, we must compute $C_1, \dots, C_n$.
From the implementation details at p.33 in LAB-01.pdf, we can generalize the equation of $C_n$ as follows:

\begin{align*}
    C_1 &= 
    \textcolor{Green}{G_0} + 
    \textcolor{orange}{C_0} \cdot \textcolor{blue}{P_0} \\
    C_2 &= 
    \textcolor{Green}{G_1} + 
    \textcolor{Green}{G_0} \cdot \textcolor{blue}{P_1} + 
    \textcolor{orange}{C_0} \cdot \textcolor{blue}{P_0} \cdot \textcolor{blue}{P_1} \\
    C_3 &= 
    \textcolor{Green}{G_2} + 
    \textcolor{Green}{G_1} \cdot \textcolor{blue}{P_2} + 
    \textcolor{Green}{G_0} \cdot \textcolor{blue}{P_1} \cdot \textcolor{blue}{P_2} + 
    \textcolor{orange}{C_0} \cdot \textcolor{blue}{P_0} \cdot \textcolor{blue}{P_1} \cdot \textcolor{blue}{P_2} \\
    \vdots \\
    C_n &= 
    \underline{\textcolor{Green}{G_{n-1}}} + 
    \underline{\textcolor{Green}{G_{n-2}} \cdot \textcolor{blue}{P_{n-1}}} + 
    \cdots + 
    \underline{\textcolor{Green}{G_0} \cdot \textcolor{blue}{P_1 \cdot \cdots \cdot P_{n-1}}} + 
    \textcolor{orange}{C_0 \cdot \textcolor{blue}{P_0 \cdot \cdots \cdot P_{n-1}}}
\end{align*}

And $C_n$ can be rewritten as:

\begin{align*}
    C_n = 
    \underline{\sum_{i=0}^{n-1} \left( \textcolor{Green}{G_i} \cdot \prod_{j=i+1}^{n-1} \textcolor{blue}{P_j} \right)} + 
    \textcolor{orange}{C_0} \cdot \prod_{j=0}^{n-1} \textcolor{blue}{P_j}
\end{align*}

Observe the equation for $C_3$, we can see that since we are only allowed to use 2-input gates,
for each term, we need:

\begin{itemize}
    \item $G_2$: \textcolor{red}{0} AND gate
    \item $G_1 \cdot P_2$: \textcolor{red}{1} AND gate
    \item $G_0 \cdot P_1 \cdot P_2$: \textcolor{red}{2} AND gates
    \item $C_0 \cdot P_0 \cdot P_1 \cdot P_2$: \textcolor{red}{3} AND gates
\end{itemize}

And we need \textcolor{red}{3} OR gates to sum up all the terms.
\bigskip

Consider the structure of a more complicated example:

\begin{figure}[H]
    \centering
    \includegraphics[width=1\textwidth]{7_binary_tree_structure.jpeg}
\end{figure}

We can see that for $n = 8$, for the term that needs the most levels, it needs $4$ levels of AND gates, 
and this happens at $C_0 \cdot \prod_{j=0}^{7} P_j$.
\bigskip

Generalizing this to the $n$-bit case, the maximal depth happens at $C_0 \cdot \prod_{j=0}^{n-1} P_j$,
and this requires $\lceil \log_2(n+1) \rceil$ levels of AND gates.
\bigskip

The next step is to OR these resulting $n+1$ terms together, and this is again a binary tree structure,
and the depth is also $\lceil \log_2(n+1) \rceil$, the process is similar and shown below:

\begin{figure}[H]
    \centering
    \includegraphics[width=0.8\textwidth]{7_OR_tree.jpeg}
\end{figure}

Last, we did not count the process of generating $G_i, P_i$ as a level, so the total depth is:

\begin{align*}
    \lceil \log_2(n+1) \rceil + \lceil \log_2(n+1) \rceil + 1 = 2 \lceil \log_2(n+1) \rceil + 1
\end{align*}

But until now, we have not considered the sum, which has the equation:

\begin{align*}
    S_i = A_i \oplus B_i \oplus C_i
\end{align*}

Thus, for each bit, we need 2 XOR gates ($A_i \oplus B_i, \quad (A_i \oplus B_i) \oplus C_i$).
Observe the following figure, we can see that we only need to compute $S_3$ by $C_3 \oplus P_3$,
so we only need up to $C_{n-1}$:

\begin{figure}[H]
    \centering
    \includegraphics[width=0.6\textwidth]{7_sum.jpg}
\end{figure}

Therefore, the sum depth is $C_{n-1}$ (which needs $2\lceil \log_2(n) \rceil + 1$ levels)
plus the two XOR levels, which is $2\lceil \log_2(n) \rceil + 3$.
\bigskip

From the same above graph, the number of levels 
would depend on whether $C_4$ or $S_3$ is having the maximal depth,
and in our generalized case, it would be:

\begin{align*}
    \max \left\{ 2 \lceil \log_2(n+1) \rceil + 1, 2 \lceil \log_2(n) \rceil + 3 \right\} \qquad \square
\end{align*}

\end{document}