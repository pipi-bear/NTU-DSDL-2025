\documentclass{article}
\usepackage[utf8]{inputenc}
\usepackage[english]{babel}
\usepackage[]{amsthm} 
\usepackage[]{amssymb} 
\usepackage{amsmath}
\usepackage{hyperref}
\usepackage{multirow}
\usepackage{tabularx}
\usepackage{graphicx}
\graphicspath{ {./solution_images/} }
\usepackage{float}
\usepackage[table]{xcolor}
\usepackage[dvipsnames]{xcolor}
\usepackage[thinc]{esdiff}


\title{Digital System Design and Lab: HW3}
\author{Lo Chun, Chou \\ R13922136}
\date\today

\begin{document}
\setlength{\parindent}{0pt}
\maketitle 

\section*{1}

By lecture slide LEC-09 p.10-11, 
we knew that using two three-state buffers with one inverter could do data selection, 
and is equivalent to a 2-to-1 MUX:

\begin{figure}[h]
    \centering
    \includegraphics[width=0.5\textwidth]{1_three_state_buffer_w_inverter.jpeg}
\end{figure}

So, we can use the two 4-to-1 MUXs and this setting to implement the 8-to-1 MUX as follows:

\begin{figure}[h]
    \centering
    \includegraphics[width=0.5\textwidth]{1_sol.jpeg}
\end{figure}

\section*{2}

First, observe that part of the given circuit is a $\bar{S}-\bar{R}$ latch:

\begin{figure}[h]
    \centering
    \includegraphics[width=0.8\textwidth]{2_Sbar_Rbar_latch.jpeg}
\end{figure}

with the inputs:

\begin{align*}
    \bar{S} &= (AB)' = A' + B' \\
    \bar{R} &= (AB')' = A' + B
\end{align*}

Thus, we can form the truth table of this latch by checking the values of $\bar{S}$, $\bar{R}$ and $Q$ in the table above:

\begin{center}
    \begin{tabular}{ |c|c||c|c|c||c| } 
        \hline
        \cline{1-6}
        A & B & $\raisebox{-0.4ex}{$\bar{S}$}$ & $\raisebox{-0.4ex}{$\bar{R}$}$ & Q & $Q^+$ \\
        \hline
        0 & 0 & 1 & 1 & 0 & 0 \\
        0 & 0 & 1 & 1 & 1 & 1 \\
        0 & 1 & 1 & 1 & 0 & 0 \\
        0 & 1 & 1 & 1 & 1 & 1 \\
        1 & 0 & 1 & 0 & 0 & 0 \\
        1 & 0 & 1 & 0 & 1 & 0 \\
        1 & 1 & 0 & 1 & 0 & 1 \\
        1 & 1 & 0 & 1 & 1 & 1 \\
        \hline
    \end{tabular}
\end{center}



\section*{3}

The following is the different cases of the latch:

\begin{figure}[H]
    \centering
    \includegraphics[width=0.7\textwidth]{3_derivation.jpeg}
\end{figure}

\subsection*{(1)}

From the above cases, we can see that when $R = 1$ and $H = 0$, $P = 1 \neq Q' = 0$.
Therefore, we should not let:

\begin{align*}
    R = 1 \quad \text{and} \quad H = 0
\end{align*}

\subsection*{(2)}

The next-state table is shown below:

\begin{center}
    \begin{tabular}{ |c|c|c||c| } 
        \hline
        \cline{1-4}
        R & H & Q & Q+ \\ 
        \hline
        0 & 0 & 0 & 0 \\ 
        0 & 0 & 1 & 0 \\ 
        0 & 1 & 0 & 0 \\ 
        0 & 1 & 1 & 1 \\ 
        1 & 0 & 0 & X \\
        1 & 0 & 1 & 1 \\ 
        1 & 1 & 0 & X \\ 
        1 & 1 & 1 & 1 \\ 
        \hline
    \end{tabular}
\end{center}

And we can construct the K-map as follows:

\begin{figure}[H]
    \centering
    \includegraphics[width=0.4\textwidth]{3_kmap.jpeg}
\end{figure}

Which would give us the characteristic equation:

\begin{align*}
    Q^+ = R + H \cdot Q
\end{align*}

\section*{4}

\subsection*{(1)}

\begin{figure}[H]
    \centering
    \includegraphics[width=1\textwidth]{4_1_clagl.png}
\end{figure}

\subsection*{(2)}

\begin{figure}[H]
    \centering
    \includegraphics[width=0.9\textwidth]{4_2_rcagl.png}
\end{figure}


\section*{5}

I use Surfer instead of GTKWave to present the waveform:

\begin{figure}[H]
    \centering
    \includegraphics[width=1\textwidth]{5_waveform.png}
\end{figure}

\section*{6}

We can find the maximum delay and one of the transition from the attached terminal output screenshots.

\subsection*{(1)}

\begin{figure}[H]
    \centering
    \includegraphics[width=0.7\textwidth]{6_(1).png}
\end{figure}

\subsection*{(2)}

\begin{figure}[H]
    \centering
    \includegraphics[width=0.7\textwidth]{6_(2).png}
\end{figure}

\section*{7}










\end{document}