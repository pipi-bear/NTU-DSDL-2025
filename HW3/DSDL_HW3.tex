\documentclass{article}
\usepackage[utf8]{inputenc}
\usepackage[english]{babel}
\usepackage[]{amsthm} 
\usepackage[]{amssymb} 
\usepackage{amsmath}
\usepackage{hyperref}
\usepackage{multirow}
\usepackage{tabularx}
\usepackage{graphicx}
\graphicspath{ {./solution_images/} }
\usepackage{float}
\usepackage[table]{xcolor}
\usepackage[dvipsnames]{xcolor}
\usepackage[thinc]{esdiff}


\title{Digital System Design and Lab: HW3}
\author{Lo Chun, Chou \\ R13922136}
\date\today

\begin{document}
\setlength{\parindent}{0pt}
\maketitle 

\section*{1}

By lecture slide LEC-09 p.10-11, 
we knew that using two three-state buffers with one inverter could do data selection, 
and is equivalent to a 2-to-1 MUX:

\begin{figure}[h]
    \centering
    \includegraphics[width=0.5\textwidth]{1_three_state_buffer_w_inverter.jpeg}
\end{figure}

So, we can use the two 4-to-1 MUXs and this setting to implement the 8-to-1 MUX as follows:

\begin{figure}[h]
    \centering
    \includegraphics[width=0.5\textwidth]{1_sol.jpeg}
\end{figure}

\section*{2}

The derivation process and the resulting truth table are shown below:

\begin{figure}[h]
    \centering
    \includegraphics[width=0.7\textwidth]{2_derivation_and_sol.jpeg}
\end{figure}


\section*{3}

The following is the different cases of the latch:

\begin{figure}[H]
    \centering
    \includegraphics[width=0.7\textwidth]{3_derivation.jpeg}
\end{figure}

\subsection*{(1)}

From the above cases, we can see that when $R = 1$ and $H = 0$, $P = 1 \neq Q' = 0$.
Therefore, we should not let:

\begin{align*}
    R = 1 \quad \text{and} \quad H = 0
\end{align*}

\subsection*{(2)}

The next-state table is shown below:

\begin{center}
    \begin{tabular}{ |c|c|c||c| } 
        \hline
        \cline{1-4}
        R & H & Q & Q+ \\ 
        \hline
        0 & 0 & 0 & 0 \\ 
        0 & 0 & 1 & 0 \\ 
        0 & 1 & 0 & 0 \\ 
        0 & 1 & 1 & 1 \\ 
        1 & 0 & 0 & X \\
        1 & 0 & 1 & 1 \\ 
        1 & 1 & 0 & X \\ 
        1 & 1 & 1 & 1 \\ 
        \hline
    \end{tabular}
\end{center}

And we can construct the K-map as follows:

\begin{figure}[H]
    \centering
    \includegraphics[width=0.4\textwidth]{3_kmap.jpeg}
\end{figure}

Which would give us the characteristic equation:

\begin{align*}
    Q^+ = R + H \cdot Q
\end{align*}



\end{document}