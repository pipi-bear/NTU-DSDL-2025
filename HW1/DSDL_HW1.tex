\documentclass{article}
\usepackage[utf8]{inputenc}
\usepackage[english]{babel}
\usepackage[]{amsthm} 
\usepackage[]{amssymb} 
\usepackage{amsmath}
\usepackage{hyperref}
\usepackage{multirow}
\usepackage{tabularx}
\usepackage[table]{xcolor}
\usepackage[dvipsnames]{xcolor}
\usepackage[thinc]{esdiff}
\newtheorem*{theorem}{Theorem}

\title{Digital System Design and Lab: HW1}
\author{Lo Chun, Chou \\ R13922136}
\date\today

\begin{document}
\setlength{\parindent}{0pt}
\maketitle 

\section*{1}

Since the following operation is correct:

\begin{align*}
    024 + 043 + 013 + 033 = 201
\end{align*}

Let base $=k \in \mathbb{Z}^+, \quad k > 4$

We can formulate the following equations:

\begin{align*}
    4 + 3 + 3 + 3 \pmod{k} &= 13 \pmod{k} = 1 \tag{1} \\
    2 + 4 + 1 + 3 + c_1 \pmod{k} &= 10 + c_1 \pmod{k} = 0 \tag{2} \\
    0 + 0 + 0 + 0 + c_2 \pmod{k} &= c_2 \pmod{k} = 2 \tag{3}
\end{align*}

Where $c_1, c_2$ are the carry-out generated by the addition of the previous digits.
\bigskip

By $(1)$, and $k > 4$, we knew that the possible values of $k$ are $6, 12$.

\subsection*{Case 1: $k = 6$}

Suppose $k = 6$, we have:

\begin{align*}
    4 + 3 + 3 + 3 \pmod{6} &= 13 \pmod{6} = 1 \tag{1} \\
    2 + 4 + 1 + 3 + c_1 \pmod{6} &= 10 + c_1 \pmod{6} = 0 \tag{2} \\
    0 + 0 + 0 + 0 + c_2 \pmod{6} &= c_2 \pmod{6} = 2 \tag{3}
\end{align*}

By $(1)$, we knew that $c_1 = 2$, thus equation $(2)$ holds since $10 + 2 = 12 \equiv 0 \pmod{6}$.
Similarly, we knew that $c_2 = 2$, thus equation $(3)$ holds since $0 + 2 = 2 \equiv 2 \pmod{6}$.
\bigskip

Therefore, $k = 6$ is a valid solution.

\subsection*{Case 2: $k = 12$}

Suppose $k = 12$, we have:

\begin{align*}
    4 + 3 + 3 + 3 \pmod{12} &= 13 \pmod{12} = 1 \tag{1} \\
    2 + 4 + 1 + 3 + c_1 \pmod{12} &= 10 + c_1 \pmod{12} = 0 \tag{2} \\
    0 + 0 + 0 + 0 + c_2 \pmod{12} &= c_2 \pmod{12} = 2 \tag{3}
\end{align*}

By $(1)$, we knew that $c_1 = 1$ since $\lfloor 13 \div 12\rfloor = 1$,
however, this contradicts with equation $(2)$ since $10 + 1 = 11 \not\equiv 0 \pmod{12}$.
\bigskip

Therefore, $k = 12$ is not a valid solution.
\bigskip

The only possible base is $k = 6$. \qed
\newpage

\section*{2}

\subsection*{(1)}

\begin{center}
    \begin{tabular}{ |c||c|c|c|c| } 
        \hline
        \multirow{2}{*}{Decimal} & \multicolumn{4}{c|}{weighted code} \\
        \cline{2-5}
         & 8 & 4 & -2 & -1 \\
        \hline
        0 & 0 & 0 & 0 & 0 \\ 
        1 & 0 & 1 & 1 & 1 \\ 
        2 & 0 & 1 & 1 & 0 \\ 
        3 & 0 & 1 & 0 & 1 \\ 
        4 & 0 & 1 & 0 & 0 \\
        5 & 1 & 0 & 1 & 1 \\ 
        6 & 1 & 0 & 1 & 0 \\ 
        7 & 1 & 0 & 0 & 1 \\ 
        8 & 1 & 0 & 0 & 0 \\
        9 & 1 & 1 & 1 & 1 \\
        \hline
    \end{tabular}
\end{center}

\subsection*{(2)}

\begin{center}
    \begin{tabular}{ |c|c||c|c|c|c| } 
        \hline
        \multirow{2}{*}{Decimal ($d$)} & \multirow{2}{*}{$9-d$} & \multicolumn{4}{c|}{weighted code for $9-d$} \\
        \cline{3-6}
         &  & 8 & 4 & -2 & -1 \\
        \hline
        0 & 9 & 1 & 1 & 1 & 1 \\ 
        1 & 8 & 1 & 0 & 0 & 0 \\ 
        2 & 7 & 1 & 0 & 0 & 1 \\ 
        3 & 6 & 1 & 0 & 1 & 0 \\ 
        4 & 5 & 1 & 0 & 1 & 1 \\ 
        5 & 4 & 0 & 1 & 0 & 0 \\ 
        6 & 3 & 0 & 1 & 0 & 1 \\ 
        7 & 2 & 0 & 1 & 1 & 0 \\ 
        8 & 1 & 0 & 1 & 1 & 1 \\
        9 & 0 & 0 & 0 & 0 & 0 \\
        \hline
    \end{tabular}
\end{center}

By obersving the table, and by the fact that $9-d$ is the complement of $d$,
we can see that the weighted code for $9-d$ is the complement of the weighted code for $d$.

\bigskip
Which means that to obtain the code of $9-d$, we could simply flip the bits of the code of $d$.


\newpage

\section*{3}

Find the complemet of the function by only DeMorgan's laws and involution law:

\begin{align*}
    F(A, B, C, D) = AB'C + (A' + B + D)(ABD' + B')
\end{align*}

As required, we can only use the below properties:

\begin{align*}
    (AB)' &= A' + B' \\
    (A + B)' &= A'B' \\
    (A')' &= A \\
\end{align*}

\begin{align*}
    F'(A, B, C, D) &= (AB'C)' \cdot \left[(A' + B + D) \cdot (ABD' + B')\right]' \\
    &= \left[\left(AB'\right)' + C' \right] \cdot \left[\left(A' + B + D\right)' + \left(ABD' + B'\right)'\right] \\
    &= \left( A' + B + C' \right) \cdot \left[\left(A' + B\right)'D' + \left(ABD'\right)'B\right] \\
    &= \left( A' + B + C' \right) \cdot \left[ AB'D' + \left(\left(AB\right)' + D\right)B\right] \\
    &= \left( A' + B + C' \right) \cdot \left[ AB'D' + \left(A' + B' + D\right)B\right] \\
    &= \left( A' + B + C' \right) \cdot \left( AB'D' + A'B + BD\right) \\
\end{align*}

\textcolor{red}{Ask is further derivation is needed.}


\newpage

\section*{4}

\newpage

\section*{5}

Obtain sum of product form:

\begin{align*}
    &(A + B + C + D)(A' + B' + C + D')(A' + C)(A + D)(B + C + D) \\
    =& \left\{\left[A + (B + C + D)\right](B + C + D)\right\}(A' + B' + C + D')(A' + C)(A + D) \\
    =& \left[A(B + C + D) + (B + C + D)(B + C + D)\right](A' + B' + C + D')(A' + C)(A + D) \\
    =& \left[AB + AC + AD + (B + C + D)\right](A' + B' + C + D')(A' + C)(A + D) \\
    =& (B + C + D)(A' + B' + C + D')(A' + C)(A + D) \\
    =& (B + C + D)\left[A' + (B' + C + D')C\right](A + D) \\
    =& (B + C + D)\left[A'D + A(B' + C + D')C\right] \\
    =& (B + C + D)\left[A'D + AB'C + AC + ACD'\right] \\
    =& (B + C + D)\left[A'D + AB'C + AC\right] \\
    =& (B + C + D)\left[A'D + AC\right] \\
    =& A'BD + A'CD + A'D + ABC + AC + ACD \\
    =& A'D + AC \qed
\end{align*}

\newpage

\section*{6}

Obtain product of sums form:

\begin{align*}
    &BCD + C'D' + B'C'D + CD \\
    =& C'D' + B'C'D + CD \\
    =& (C' + D)\textcolor{Green}{(C + D')} + B'\textcolor{Green}{(C + D')'} \\
    =& \left[(C' + D) + B'\right]\left[(C + D')' + (C' + D)\right] \\
    =& \left( B' + C' + D\right)\left( C'D + C' + D \right) \\
    =& (B' + C' + D)(C' + D) \\
    =& B'(C' + D) + (C' + D)(C' + D) \\
    =& B'(C' + D) + (C' + D) \\
    =& C' + D
    \qed
\end{align*}

\newpage

\section*{7}

\begin{center}
    \begin{tabular}{ |c|c|c||c| } 
        \hline
        \multicolumn{3}{|c|}{Inputs} & \multirow{2}{*}{Output} \\
        \cline{1-3}
        A & B & C & \\ %
        \hline
        0 & 0 & 0 & 0 \\ 
        1 & 0 & 0 & 0 \\ 
        0 & 1 & 0 & 0 \\ 
        0 & 0 & 1 & 0 \\ 
        \rowcolor{green}
        1 & 1 & 0 & 1 \\
        \rowcolor{green}
        0 & 1 & 1 & 1 \\ 
        \rowcolor{green}
        1 & 0 & 1 & 1 \\ 
        \rowcolor{green}
        1 & 1 & 1 & 1 \\ 
        \hline
    \end{tabular}
\end{center}

From the table we can form the sum of products expression:

\begin{align*}
    &ABC' + A'BC + AB'C + ABC \\
    =& ABC' + AB'C + BC \qed
\end{align*}

\end{document}