\documentclass{article}
\usepackage[utf8]{inputenc}
\usepackage[english]{babel}
\usepackage[]{amsthm} 
\usepackage[]{amssymb} 
\usepackage{amsmath}
\usepackage{multirow}
\usepackage{tabularx}
\usepackage{xcolor}
\usepackage{mathtools}
\usepackage{tcolorbox}
\usepackage[table]{xcolor}
\usepackage[dvipsnames]{xcolor}
\usepackage[thinc]{esdiff}
\newtheorem*{theorem}{Theorem}

\title{Boolean Algebra Properties}
\author{Lo Chun, Chou \\ R13922136}
\date\today

\begin{document}
\setlength{\parindent}{0pt}
\maketitle 

\newtheorem*{exmp}{Example}

\tcbset{
    greenbox/.style={colback=SpringGreen!20, colframe=SpringGreen!80, sharp corners},
    bluebox/.style={colback=SkyBlue!20, colframe=SkyBlue!80, sharp corners},
    yellowbox/.style={colback=yellow!10, colframe=yellow!80, sharp corners}
}

\section*{2.4 Basic Theorems}

\begin{tcolorbox}[greenbox, title=Idempotent Laws, coltitle=black]
    {\begin{align*}
        X + X &= X \\
        X \cdot X &= X
    \end{align*}
    }
\end{tcolorbox}

\begin{tcolorbox}[greenbox, title=Involution Law, coltitle=black]
    {\begin{align*}
        (X')' &= X
    \end{align*}
    }
\end{tcolorbox}

\begin{tcolorbox}[greenbox, title=Laws of Complementarity, coltitle=black]
    {\begin{align*}
        X + X' &= 1 \\
        X \cdot X' &= 0
    \end{align*}
    }
\end{tcolorbox}

\section*{2.5 Commutative, Associative, Distributive and DeMorgan's Laws}

\begin{tcolorbox}[greenbox, title=Distributive Laws, coltitle=black]
    {\begin{align*}
        X(Y + Z) &= XY + XZ \\
        X + YZ &= (X + Y)(X + Z)
    \end{align*}
    }
\end{tcolorbox}

\begin{exmp}
    \begin{align*}
        (\textcolor{Green}{A + B} + C')(\textcolor{Green}{A + B} + D)
        = \textcolor{Green}{A + B} + C'D 
    \end{align*} 
\end{exmp}

\begin{proof}
    \begin{align*}
        (\textcolor{Green}{X + Y})(X + Z) 
        &= \textcolor{Green}{X}(X + Z) + \textcolor{Green}{Y}(X + Z) \\
        &= XX + XZ + XY + YZ \\
        &= X + XZ + XY + YZ \\
        &= X \textcolor{orange}{\cdot 1} + XZ + XY + YZ \\
        &= X(1 + Z + Y) + YZ \\
        &= X \cdot 1 + YZ \\
        &= X + YZ 
    \end{align*}
\end{proof}

\begin{tcolorbox}[greenbox, title=DeMorgan's Laws, coltitle=black]
    {\begin{align*}
        (X + Y)' &= X' \cdot Y' \\
        (X \cdot Y)' &= X' + Y'
    \end{align*}
    }
\end{tcolorbox}

\section*{2.6 Simplification Theorems}

\begin{tcolorbox}[greenbox, title=Uniting, coltitle=black]
    {
    \begin{align*}
        XY + XY' &= X \\
        (X + Y)(X + Y') &= X
    \end{align*}
    }
\end{tcolorbox}

\begin{proof}
    \begin{align*}
        XY + XY' &= X(Y + Y') \\
        &= X \cdot 1 \\
        &= X
    \end{align*}
\end{proof}

\begin{proof}
    \begin{align*}
        (X + Y)(X + Y') &= XX + XY' + XY + YY' \\
        &= X + XY' + XY \\
        &= X(1 + Y' + Y) \\
        &= X \cdot 1 \\
        &= X
    \end{align*}
\end{proof}

\begin{tcolorbox}[greenbox, title=Absorption, coltitle=black]
    {
    \begin{align*}
        X + XY &= X \\
        X(X + Y) &= X
    \end{align*}
    }
\end{tcolorbox}

\begin{proof}
    \begin{align*}
        X(X + Y) &= (X \textcolor{orange}{+ \ 0})(X + Y) \qquad \text{(distributive)}\\
        &= X + 0 \cdot Y \\
        &= X
    \end{align*}
\end{proof}

\begin{tcolorbox}[greenbox, title=Elimination, coltitle=black]
    {
    \begin{align*}
        X + X'Y &= X + Y \\
        X(X' + Y) &= XY
    \end{align*}
    }
\end{tcolorbox}

\begin{proof}
By distributive laws, we have $X + YZ = (X + Y)(X + Z)$
\bigskip

Thus,
\begin{align*}
    X + X'Y &= (X + X')(X + Y) \\
    &= 1 \cdot (X + Y) \\
    &= X + Y
\end{align*}
\end{proof}

\begin{tcolorbox}[greenbox, title=Consensus, coltitle=black]
    {
    \begin{align*}
        XY + X'Z + \textcolor{red}{YZ} &= XY + X'Z \\
        (X + Y)(X' + Z)\textcolor{red}{(Y + Z)} &= (X + Y)(X' + Z)
    \end{align*}
    }
\end{tcolorbox}

\begin{proof}
    \begin{align*}
        XY + X'Z + YZ 
        &= XY + X'Z + \textcolor{orange}{1} \cdot YZ \\
        &= XY + X'Z + \textcolor{orange}{(X + X')}YZ \\
        &= \textcolor{red}{XY} + X'Z + \textcolor{red}{XYZ}+ X'YZ \qquad \text{(absorption)}\\
        &= XY + \textcolor{red}{X'Z + X'YZ} \qquad \text{(absorption)}\\
        &= XY + X'Z
    \end{align*}
\end{proof}

\section*{2.8 Complementing Boolean Expressions}

\begin{tcolorbox}[greenbox]
    {
    \begin{align*}
        (X_1 + X_2 + \cdots + X_n)' &= X_1' \cdot X_2' \cdots X_n' \\
        (X_1 \cdot X_2 \cdots X_n)' &= X_1' + X_2' + \cdots + X_n'
    \end{align*}
    }
\end{tcolorbox}

\section*{3.2 Exclusive-OR and Equivalence Operations}

\begin{tcolorbox}[greenbox, title=Exclusive-OR, coltitle=black]
    {
    \begin{align*}
        X \oplus Y &= XY' + X'Y
    \end{align*}
    }
\end{tcolorbox}

Because $X$ exclusive-OR $Y$ is $1$ only when $X = 1$ and $Y = 0$ ($XY'$) or $X = 0$ and $Y = 1$ ($X'Y$).
\bigskip

Another way to think:

$X$ exclusive-OR $Y$ is $1$ only when $X = 1$ or $Y = 1$, and $X, Y$ not both $1$, therefore:

\begin{align*}
    X \oplus Y 
    &= (X + Y)(XY)' \\
    &= (X + Y)(X' + Y') \\
    &= XY' + X'Y
\end{align*}

\begin{tcolorbox}[greenbox, title=Properties of Exclusive-OR, coltitle=black]
    {
    \begin{align*}
        &X \oplus 0 = X \\
        &X \oplus 1 = X' \\
        &X \oplus X     = 0 \\
        &X \oplus X' = 1 \\
        &X \oplus Y = Y \oplus X \ \text{(commutative)}\\
        &X \oplus (Y \oplus Z) = (X \oplus Y) \oplus Z  = X \oplus Y \oplus Z \ \text{(associative)}\\
        &X(Y \oplus Z) = XY \oplus XZ \ \text{(distributive)}\\
        &\textcolor{orange}{(X \oplus Y)' = X \oplus Y'  = X' \oplus Y = XY + X'Y'}\\
    \end{align*}
    }
\end{tcolorbox}

\begin{proof}
    \begin{align*}
        (X \oplus Y)'
        &= (XY' + X'Y)' \\
        &= (XY')' \cdot (X'Y)' \\
        &= (X' + Y)(X + Y') \\
        &= X'Y' + XY 
    \end{align*}
\end{proof}

\begin{tcolorbox}[greenbox, title=Equivalence Relation, coltitle=black]
    {
    \begin{align*}
        &(X \equiv Y)  = 1 \iff X = Y \\
        \Rightarrow \ & (X \equiv Y) = XY + X'Y'
    \end{align*}
    }
\end{tcolorbox}

Because $X \equiv Y$ is $1$ only when $X = Y = 1$ or $X = Y = 0$.
\bigskip

Therefore, combining the above equations, we have:

\begin{tcolorbox}[yellowbox, title=Equivalence is the complement of exclusive-OR, coltitle=black]
    {
    \begin{align*}
        (X \equiv Y) = XY + X'Y' = (X \oplus Y)'
    \end{align*}
    }
\end{tcolorbox}

Hence, \textcolor{orange}{the equivalence gate is also called the exclusive-NOR gate.}

Also, note that:

\begin{align*}
    &X \oplus Y = XY' + X'Y \\
    \Rightarrow \ & (X \oplus Y)' = \textcolor{Green}{(XY' + X'Y)'  = XY + X'Y'}
\end{align*}
m 
This is useful, for example:

\begin{align*}
    A' \oplus B \oplus C
    &= (A'B' + AB) \oplus C \\
    &= \textcolor{orange}{(A'B' + AB)'}C + (A'B' + AB)C' \\
    &= \textcolor{orange}{(A'B + AB')}C + (A'B' + AB)C' \\
    &= A'BC + AB'C + A'B'C' + ABC'
\end{align*}

\end{document}